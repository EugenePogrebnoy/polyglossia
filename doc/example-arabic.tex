\documentclass[a4paper]{book}%
\usepackage[no-math]{fontspec}
\usepackage{xltxtra,url,amsmath}
\usepackage[quiet]{polyglossia}
\setdefaultlanguage[calendar=gregorian,hijricorrection=1,locale=mashriq]{arabic}
\setotherlanguage[variant=british]{english}
\setotherlanguage{farsi}
\defaultfontfeatures{Scale=MatchLowercase}
\newfontfamily\arabicfont[Script=Arabic,Scale=1.3]{Scheherazade}%
\newfontfamily\arabicfonttt[Script=Arabic,Scale=.75]{DejaVu Sans Mono}
\newfontfamily\farsifont[Script=Arabic,Scale=1.1,WordSpace=2]{IranNastaliq}
\parindent 0pt
\makeatletter
\makeatother
\title{امتحان تأييد اللغة العربية}
\author{فرانسوا شاريت}
\begin{document}
\pagenumbering{alph}
\maketitle
\tableofcontents
\chapter{امتحان}
\pagenumbering{arabic}
\section{لغات مختلفة}


\textbf{العربية}\footnote{%
من «\LR{\textenglish{\url{http://ar.wikipedia.org/wiki/}}\RL{\ttfamily لغة عربية}}»} 
أكبر لغات المجموعة السامية من حيث عدد المتحدثين، وإحدى أكثر اللغات انتشارا في
العالم، يتحدثها أكثر من ٤٢٢ مليون نسمة،١ ويتوزع متحدثوها في المنطقة المعروفة
باسم الوطن العربي، بالإضافة إلى العديد من المناطق الأخرى المجاورة كالأحواز وتركيا
وتشاد ومالي والسنغال. وللغة العربية أهمية قصوى لدى أتباع الديانة الإسلامية، فهي
لغة مصدري التشريع الأساسيين في الإسلام: القرآن، والأحاديث النبوية المروية عن النبي
محمد، ولا تتم الصلاة في الإسلام (وعبادات أخرى) إلا بإتقان بعض من كلمات هذه اللغة.
والعربية هي أيضاً لغة طقسية رئيسية لدى عدد من الكنائس المسيحية في العالم العربي،
كما كتبت بها الكثير من أهم الأعمال الدينية والفكرية اليهودية في العصور الوسطى.
وإثر انتشار الإسلام، وتأسيسه دولا، ارتفعت مكانة اللغة العربية، وأصبحت لغة السياسة
والعلم والأدب لقرون طويلة في الأراضي التي حكمها المسلمون، وأثرت العربية، تأثيرا
مباشرا أو غير مباشر على كثير من اللغات الأخرى في العالم الإسلامي، كالتركية
والفارسية والأردية مثلا.

\textfarsi{\bfseries فارسی}\footnote{%
از «\LR{\textenglish{\url{http://fa.wikipedia.org/wiki/}}\RL{\ttfamily فارسي}}»}
\begin{farsi}
یا پارسی، (که دری، فارسی دری، و پارسی دری نیز نامیده می‌شود) زبانی است که
در کشورهای ایران، افغانستان، تاجیکستان و ازبکستان به آن سخن می‌رانند.
(برخی زبان فارسی در تاجیکستان و ازبکستان و چین را فارسی تاجیکی نام
می‌گذارند).  
\end{farsi}

\newpage
\begin{english}
\textbf{English}\footnote{%
	From \url{http://en.wikipedia.org/wiki/English_language}} 
is a West Germanic language originating in England, and the first language for
most people in Australia, Canada, the Commonwealth Caribbean, Ireland, New
Zealand, the United Kingdom and the United States of America (also commonly
known as the Anglosphere). It is used extensively as a second language and as
an official language throughout the world, especially in Commonwealth countries
and in many international organisations.

\textarabic{١ ٢ ٣}

\end{english}
\clearpage

\section{أعمال تأريخية \textenglish{(Calendar operations)}}


\textenglish{\today} = \LR{\today} = \today\ = \Hijritoday\footnote{ 
	محسوب بـ \textenglish{\textsf{hijrical.sty}}} 

%\newpage
\subsection{فلان}
\textenglish{This is English: a b c}\marginpar{انكليزي} %FIXME! cf farsitex?

\subsubsection{فلان فلان}
\begin{enumerate}
	\item مثال
	\item مثال
		\begin{enumerate}
			\item مثال
			\item مثال
		\end{enumerate}

	\item مثال	
\end{enumerate}

\begin{table}[h]
	\centering
	\begin{tabular}{cc}
		ا & ب  \\
		ج & د  
	\end{tabular}
	\caption{هذا المثال}
\end{table}

\[
x^\text{مال مال}
\]

\begin{equation}
	x^2 + y^2 = z^2
	\label{test}
\end{equation}
\end{document}
